\documentclass{article}
\usepackage{amsmath}

\title{Trabajo Práctico Nº 2 - Opción Simplificada}
\author{Profesor: Lucas Trejo}
\date{}

\begin{document}

\maketitle

\section*{Actividades}

\begin{enumerate}
    \item Ver con atención los siguientes videos:
    \begin{enumerate}
        \item \url{https://youtu.be/qjX4wKUoK7E}
        \item \url{https://youtu.be/uPbE524NcWM}
    \end{enumerate}
    (En el primer video se desarrolla una explicación más detallada, el segundo solo muestra algunos ejemplos resueltos)

    \item Escribir cada uno de los siguientes números en notación científica:
    \begin{itemize}
        \item[a)] 12 000 000 = 
        \item[b)] 44 000 000 = 
        \item[c)] 7 000 000 000 = 
        \item[d)] 1 500 000 000 000 = 
        \item[e)] 200 = 
        \item[f)] 5 = 
    \end{itemize}

    \item Escribir como número decimal y en notación científica:
    \begin{itemize}
        \item[a)] Doce millones 
        \item[b)] Un billón 
        \item[c)] Cuatrocientos millones 
        \item[d)] Siete mil cuatrocientos millones
        \item[e)] Dos mil millones 
        \item[f)] Un trillón 
    \end{itemize}

    \item Expresar en notación científica:
    \begin{itemize}
        \item[a)] 0,05 = 
        \item[b)] 0,0018 = 
        \item[c)] 0,000085 = 
        \item[d)] 0,000000186 = 
        \item[e)] 0,000000000024 = 
        \item[f)] 0,000000000000001 = 
    \end{itemize}

    \item Indicar cuáles de estos números no están escritos en notación científica, justificando el motivo:
    \begin{itemize}
        \item[a)] $9,81 \cdot 10^{-4}$
        \item[b)] $35 \cdot 10^8$
        \item[c)] $1,5 \cdot 10^{2,5}$
        \item[d)] $0,981 \cdot 10^{14}$
        \item[e)] $1,6 \cdot 34$
        \item[f)] $5 \cdot 10^0$
        \item[g)] $3,5 \cdot 10^{\sqrt{3}}$
        \item[h)] $9,99 \cdot 10^{-23}$
    \end{itemize}

    \item Investigar y responder las siguientes preguntas, expresando los resultados en notación convencional y científica (redondear a 3 cifras significativas):
    \begin{itemize}
        \item[a)] ¿Cuál es la distancia mínima, en kilómetros, entre el Sol y la Luna?
        \item[b)] ¿Cuál es la distancia mínima, en kilómetros, entre la Tierra y el Sol?
        \item[c)] ¿Cuántos metros recorre la luz en un segundo?
        \item[d)] ¿Cuántos kilómetros recorre la luz en un año?
        \item[e)] ¿Cuántos kilómetros tiene la circunferencia de la Tierra?
        \item[f)] ¿Cuántas estrellas se estima que hay en nuestra galaxia, la Vía Láctea?
        \item[g)] ¿Cuántas neuronas tiene el cerebro humano?
        \item[h)] ¿Cuál es la población mundial en la actualidad?
        \item[i)] ¿Hace cuántos años se calcula que se formó nuestro planeta, la Tierra?
        \item[j)] ¿Hace cuántos años se extinguieron los dinosaurios?
        \item[k)] ¿Cuántas bacterias calculan los científicos que viven dentro de nuestro cuerpo?
        \item[l)] ¿Cuál es el diámetro aproximado de una bacteria (en metros)?
        \item[m)] ¿Cuál es el tamaño del virus más grande que se conoce (en metros)?
        \item[n)] ¿Cuál es el monto total, en dólares, de la deuda externa argentina?
        \item[o)] ¿Cuánto dinero se invierte en presupuesto militar cada año en EEUU?
        \item[p)] ¿Cuál es el total del gasto militar a nivel mundial?
    \end{itemize}

    \item Escribir al menos tres preguntas cuya respuesta sea un número muy grande.
    
    \item Escribir al menos tres preguntas cuya respuesta sea un número muy pequeño.
\end{enumerate}

\end{document}
