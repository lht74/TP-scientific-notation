\documentclass[12pt]{exam}
\usepackage[spanish]{babel}
\usepackage{hyperref}
\usepackage{amsmath}
\usepackage{multicol}
\usepackage[left=10mm, right= 20mm, bottom= 20mm, top= 35mm, headsep=10mm, headheight=0mm]{geometry}
\usepackage{setspace}
\usepackage{graphicx}
\renewcommand\partlabel{\thepartno)}

%%% DEFINICIONES PARA MODIFICAR APARIENCIA DEL ENCABEZADO%%%
\usepackage{etoolbox}
\makeatletter
\patchcmd{\@fullhead}{\hrule}{\hrule\vskip2pt\hrule height 2pt}{}{}
\patchcmd{\run@fullhead}{\hrule}{\hrule\vskip2pt\hrule height 2pt}{}{}
\makeatother
\pagestyle{headandfoot}
\firstpageheadrule
\firstpageheader {\escuela \\ Curso: \curso}
                 {MATEMÁTICA \\ \tpnombre}
                  {\lugar \\ \today}  
\firstpagefooter{Prof: \textit{Lucas H. Trejo}}{}{\thepage}
\runningfooter{Prof: \textit{Lucas H. Trejo}}{}{Pag. \thepage\ of \numpages}

%%%%%%%%%%%%%%%%%%%%%%%%%%%%%%%%%%%%%%%%%%%%%%%%%%%%%%%%%%%%%%%%%%%%%%%%%%%%%%%
                                       %%% REEMPLAZAR LOS PARÁMETROS AQUÍ    %%  
\newcommand{\escuela}{EES Nº 32}                                             %%             
\newcommand{\curso}{  5º3ª}                                                  %%             
\newcommand{\tpnombre}{TP Notación Científica}                     %%
\newcommand{\lugar}{La Plata}                                               %%
%%%%%%%%%%%%%%%%%%%%%%%%%%%%%%%%%%%%%%%%%%%%%%%%%%%%%%%%%%%%%%%%%%%%%%%%%%%%%%%


\begin{document}

\begin{questions}

    \question Ver con atención los siguientes videos:
     
         \url{https://youtu.be/qjX4wKUoK7E} \\
         \url{https://youtu.be/uPbE524NcWM}
    
    (En el primer video se desarrolla una explicación más detallada, el segundo solo muestra algunos ejemplos resueltos).

    \question Escribir cada uno de los siguientes números en notación científica:
    \begin{parts}
        \begin{multicols}{2}
            
        \part 12 000 000 = 
        \part 44 000 000 = 
        \part 7 000 000 000 = 
        \part 1 500 000 000 000 = 
        \part 200 = 
        \part 5 = 
        \end{multicols}
    \end{parts}


    \question Escribir como número decimal y en notación científica:
    \begin{parts}
        \begin{multicols}{2}
            
        
        \part Doce millones
        \part Un billón
        \part Cuatrocientos millones
        \part Siete mil cuatrocientos millones
        \part Dos mil millones
        \part Un trillón
    \end{multicols}
    \end{parts}

    \question Expresar en notación científica:
    \begin{parts}
        \begin{multicols}{2}
            
        
        \part 0,05 = 
        \part 0,0018 = 
        \part 0,000085 = 
        \part 0,000000186 = 
        \part 0,000000000024 = 
        \part 0,000000000000001 = 
    \end{multicols}
    \end{parts}

    \question Indicar cuáles de estos números no están escritos en notación científica, justificando el motivo:
    \begin{parts}
        \begin{multicols}{2}         
        
        \part $9,81 \cdot 10^{-4}$
        \part $35 \cdot 10^8$
        \part $1,5 \cdot 10^{2,5}$
        \part $0,981 \cdot 10^{14}$
        \part $1,6 \cdot 34$
        \part $5 \cdot 10^0$
        \part $3,5 \cdot 10^{\sqrt{3}}$
        \part $9,99 \cdot 10^{-23}$
    \end{multicols}
    \end{parts}
    \setstretch{1.5}
    \question Investigar y responder las siguientes preguntas, expresando los resultados en notación convencional y científica (redondear a 3 cifras significativas):
    \begin{parts}
        \part ¿Cuál es la distancia mínima, en kilómetros, entre el Sol y la Luna?
        \part ¿Cuál es la distancia mínima, en kilómetros, entre la Tierra y el Sol?
        \part ¿Cuántos metros recorre la luz en un segundo?
        \part ¿Cuántos kilómetros recorre la luz en un año?
        \part ¿Cuántos kilómetros tiene la circunferencia de la Tierra?
        \part ¿Cuántas estrellas se estima que hay en nuestra galaxia, la Vía Láctea?
        \part ¿Cuántas neuronas tiene el cerebro humano?
        \part ¿Cuál es la población mundial en la actualidad?
        \part ¿Hace cuántos años se calcula que se formó nuestro planeta, la Tierra?
        \part ¿Hace cuántos años se extinguieron los dinosaurios?
        \part ¿Cuántas bacterias calculan los científicos que viven dentro de nuestro cuerpo?
        \part ¿Cuál es el diámetro aproximado de una bacteria (en metros)?
        \part ¿Cuál es el tamaño del virus más grande que se conoce (en metros)?
        \part ¿Cuál es el monto total, en dólares, de la deuda externa argentina?
        \part ¿Cuánto dinero se invierte en presupuesto militar cada año en EEUU?
        \part ¿Cuál es el total del gasto militar a nivel mundial?
    \end{parts}

    \question Escribe al menos tres preguntas cuya respuesta sea un número muy grande.

    \question Escribe al menos tres preguntas cuya respuesta sea un número muy pequeño.

\end{questions}

\end{document}
